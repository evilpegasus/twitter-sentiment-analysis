% !TEX TS-program = pdflatex
% !TEX encoding = UTF-8 Unicode

% This is a simple template for a LaTeX document using the "article" class.
% See "book", "report", "letter" for other types of document.

\documentclass[11pt]{article} % use larger type; default would be 10pt

\usepackage[utf8]{inputenc} % set input encoding (not needed with XeLaTeX)

%%% Examples of Article customizations
% These packages are optional, depending whether you want the features they provide.
% See the LaTeX Companion or other references for full information.

%%% PAGE DIMENSIONS
\usepackage{geometry} % to change the page dimensions
\geometry{a4paper} % or letterpaper (US) or a5paper or....
% \geometry{margin=2in} % for example, change the margins to 2 inches all round
% \geometry{landscape} % set up the page for landscape
%   read geometry.pdf for detailed page layout information

\usepackage{graphicx} % support the \includegraphics command and options

% \usepackage[parfill]{parskip} % Activate to begin paragraphs with an empty line rather than an indent

%%% PACKAGES
\usepackage{booktabs} % for much better looking tables
\usepackage{array} % for better arrays (eg matrices) in maths
\usepackage{paralist} % very flexible & customisable lists (eg. enumerate/itemize, etc.)
\usepackage{verbatim} % adds environment for commenting out blocks of text & for better verbatim
\usepackage{subfig} % make it possible to include more than one captioned figure/table in a single float
% These packages are all incorporated in the memoir class to one degree or another...

%%% HEADERS & FOOTERS
\usepackage{fancyhdr} % This should be set AFTER setting up the page geometry
\pagestyle{fancy} % options: empty , plain , fancy
\renewcommand{\headrulewidth}{0pt} % customise the layout...
\lhead{}\chead{}\rhead{}
\lfoot{}\cfoot{\thepage}\rfoot{}

%%% SECTION TITLE APPEARANCE
\usepackage{sectsty}
\allsectionsfont{\sffamily\mdseries\upshape} % (See the fntguide.pdf for font help)
% (This matches ConTeXt defaults)

%%% ToC (table of contents) APPEARANCE
\usepackage[nottoc,notlof,notlot]{tocbibind} % Put the bibliography in the ToC
\usepackage[titles,subfigure]{tocloft} % Alter the style of the Table of Contents
\renewcommand{\cftsecfont}{\rmfamily\mdseries\upshape}
\renewcommand{\cftsecpagefont}{\rmfamily\mdseries\upshape} % No bold!

%%% END Article customizations

%%% The "real" document content comes below...

\title{A Linguistic Analysis of Twitter Tweets for Donald J. Trump}
\author{Ming Fong\\
University of California, Berkeley\\
Linguistics 55AC}
%\date{} % Activate to display a given date or no date (if empty),
         % otherwise the current date is printed 

\begin{document}
\maketitle

\section{Introduction}

In the past decade, politicians, business leaders, and celebrities have increasingly turned to social media to reach a wider audience.
Of the many platforms available, Twitter has become one of the most popular social media outlets for those looking to reach a large follower base.
Among Twitter's most influential users is President Donald J. Trump.
President Trump's use of Twitter has ranged from attacks against political opponents to updates on actions he has taken as President of the United States.
The President's frequent use of Twitter is unprecedented among American leaders and has led to criticism from opponents and praise from supporters.
In this paper, we will explore linguistic trends of President Trump's Twitter posts with respect to the November 3, 2020 US Presidential election.

\section{Methodology}

We will collect Twitter Tweet text and the associated metadata from Tweets sent by from the account of Donald J. Trump (@realDonaldTrump).
Using that data, we can perform analyses on the various types of data we collect.

\subsection{Data Collection}

Our data will be directly sourced from Twitter using the Twitter Developer API.
We will read the tweets into a Pandas Dataframe with the columns
$$"id", "date", "text"$$
Due to limitations with the Twitter Developer API, our dataset is capped at the 3200 most recent tweets.
The timeframe of our data ranges from 14 September, 2020 to 11 December, 2020.

\subsection{Data Analysis}

Before performing any analysis of the tweet text data, we must first clean the text.
Any non-alphanumeric characters and elements such as links are removed.
Additionally, all retweets are removed since their text were not written by President Trump.


Using the text of the tweets, we can assign a sentiment score to each tweet.
We do this using $textblob$, a Python module for text processing.


\end{document}
